\documentclass[../main.tex]{subfiles}

\begin{document}

\section{Limitations}

The greatest limitation with the proposed approach to SER is the lack of 
a large and diverse dataset to train on. For instance, the TESS dataset contains 2800 
audio data given by two female actors. Although the CREMA-D dataset does provide 
more diversity with with more 91 actors, it isn't enough for the task of general 
SER. The use of audio data augmentation could solve this issue somewhat, however, 
the real world has a variety of vocal nuances that are difficult to replicate 
with augmentation. A more ideal solution would be to 
include a more diverse range actors. However, gathering such data and handling 
its computational demands is challenging. Even if we were to overcome these 
challenges, the model would still be limited to adult English speakers.

Moreover, the model's complexity increases overfitting risks, especially with 
small datasets. While dropout layers and early stopping can help, simpler models, 
like a CNN with attention mechanisms, may be better suited for smaller datasets 
like TESS.

\end{document}