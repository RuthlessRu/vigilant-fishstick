\documentclass[../main.tex]{subfiles}

\begin{document}

\section{TODO: Limitations}

% Describe some settings in which we’d expect your
% approach to perform poorly, or where all existing
% models fail. Try to guess or explain why these
% limitations are the way they are. Give some examples
% of possible extensions, ways to address these
% limitations or open problems


The greatest limitation with the proposed approach to SER is the lack of 
a large and diverse dataset to train on. For instance, the TESS dataset contains 2800 
audio data given by two female actors while the RAVDESS dataset contains
1440 audio data given by 24 actors all from North America. Although the
CREMA-D dataset does provide more diversity with 7442 audio data and 91 actors, 
it isn't enough for the task of general SER. The use of audio data augmentation
techniques mentioned before could help with alleviating this issue to 
some degree, however, the real world has a variety of vocal nuances that
are difficult to replicate with augmentation. A more ideal solution would be to 
include a more diverse range actors. 

Moreover, the model is trained on adult English speakers, so it is likely to 
learn features not relevant to SER. This can lead poor performance with 
non-English speakers. For example, the model may misclassify non-English speakers 
as angry when they are in fact neutral. Again, an ideal solution would be to 
include more diversity if the model is to be used in a real-world setting. 
However, not only would it be difficult to collect more data but also the 
computational resources required to train tens of thousands of audio data 
is not always feasible, especially if audio is converted to spectrograms.

Another limitation stems from the complexity of the model architecture. A complex 
model has a higher risk of overfitting, especially when the dataset is 
relatively small. This can be mitigated by using dropout layers and early stopping
as proposed. However, if the dataset can only be as small as the TESS dataset, then 
a simpler model, such as CNN with an attention mechanism, may be more appropriate.

\end{document}