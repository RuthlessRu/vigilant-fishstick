\documentclass[../main.tex]{subfiles}

\begin{document}

\section{Background and Related Works} 
Speech signals are the natural medium of 
human communication that convey information about the speaker's message, 
perspective and identity as well as their emotions. Speech Emotion Recognition 
(SER) is a classification problem that aims to infer the emotional state of a 
speaker from speech signals. Although recognizing emotions is easy for humans, 
it is a challenging task for machines. This is because speech itself is a 
complex signal that differs from person to person and can change depending on 
context and intonation \citep{Hashem2023} \citep{Koduru2020}. In recent years, there have been successful attempts at 
using Deep Neural Networks (DNN) and Convolutional Neural Networks (CNN) to 
recognize emotions from speech \citep{Pham2023} \cite{Zhang2018}. These attempts have also stressed the importance 
of finding meaningful features through feature extraction in SER.

\subsection{Feature Extraction for SER} 
Emotions in speech can be expressed 
through subtle variations in pitch, tone, energy, and timing. The goal of this 
step is to find features that can be extracted from input audio to differentiate 
between different emotions. Previous works have made use of a combination of 
acoustic features, which include prosodic and spectral features, for SER 
classification. Language information may be used as well but is often paired 
with acoustic features \citep{Pham2023} \cite{Zhang2018}. 

Prosodic features are features that are related to 
pitch, loudness, rhythm, tempo, and intonation, which give clues about how 
something is said rather than what is said. These features provide a way to 
differentiate emotions of different arousal (intensity) levels, such as happy 
and sad. However, they can't differentiate between emotions of the same arousal 
level but different valence (positivity), such as happy and angry. Spectral 
features are features that capture frequency of speech signals. The most common 
of these features are Mel-Frequency Cepstral Coefficients (MFCCs) \citep{Hashem2023}. Although 
these handcrafted features seem meaningful, they might not be enough for SER \citep{Koduru2020}. 
In this case, deep learning can be trained on Mel-spectrograms.

A Mel-spectrogram is a feature that provides a way to visually reprsent changes in loundless 
and frequency in a speech signal over time. The finer details of this
feature aren't necessary to know but can provide useful context. The Mel part of the name comes 
from scaling frequency in a way that matches the human auditory perception. 
The spectrogram part means that different parts a speech 
signal input is mapped to from a time domain to a frequency domain using 
fast Fourier transform before being stacked together. The result of stacking is 
then a graph that maps time-frequency to loudness \citep{Roberts2020}. 

\subsection{CRNN for SER} 
Most SER models use deep learning as they provide 
better results than traditional machine learning models and can learn features 
from complex raw data. These models often include Convolutional Neural Networks 
(CNN) and Recurrent Neural Networks (RNN) or a combination of both \citep{Hashem2023}. CNNs are 
effective at capturing important patterns that occur in Mel-spectrograms. 
Recurrent Neural Networks (RNN) , specifically Long Short-Term Memory (LSTM) 
networks are effective at handling sequential input data such as audio.

\subsection{Attention Mechanism} 
Not all features of the speech signal might 
be relevant in recognizing emotions \citep{Hashem2023}. For example, a speaker might sound neutral 
in the beginning and ending of a sentence but sound angry in the middle. The 
classifier should focus on the features that occur in the middle of the 
sequence. This leads to using attention mechanisms to improve the performance 
of LSTM networks by putting more emphasis on certain parts of the input 
sequence.

\end{document}