\documentclass[../main.tex]{subfiles}

\begin{document}

\section{Ethical Considerations}
In utilizing the Toronto Emotional Speech Set (TESS), we acknowledge several 
important ethical considerations regarding privacy, bias, and potential 
applications. While the dataset includes proper consent from the two female 
actors and is publicly available, we recognize that voice data is inherently 
personal and requires careful handling. The dataset has notable representation 
limitations: it only includes female voices, is limited to two age groups, and 
contains only English recordings from Canadian speakers, potentially affecting 
our model's generalization capabilities. Additionally, since the emotions are 
performed rather than naturally occurring, this may limit the applicability of 
our results in the real world. We also acknowledge that emotion recognition 
technology could potentially be misused for unauthorized surveillance or biased 
decision-making. To address these concerns, we commit to the following: 

\begin{enumerate}
    \item Bias Mitigation: Recognizing the dataset's demographic limitations, we aim to 
    use data augmentation techniques such as noise injection, time stretching, 
    and pitch  to simulate a broader range of voices. However, 
    we will be transparent about the limitations that these techniques impose, as 
    simulated diversity cannot truly replace real-world diversity in voice samples.

    \item Fairness Evaluation: We will, to the best of our ability, assess the model's 
    performance across various demographic and linguistic settings, using external 
    datasets, if possible.

    \item Usage and Transparency: We commit to using the data solely for emotion 
    classification research, openly addressing the challenges of using acted 
    emotion data and single-gender samples.    
\end{enumerate}

\end{document}