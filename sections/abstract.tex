\documentclass[../main.tex]{subfiles}

\begin{document}

\begin{abstract}
Our project focuses on building a deep learning model to classify human speech
into emotions like happiness, sadness, anger, and surprise. This could
potentially allow other people to build programs such as virtual assistants
or music apps that adjust their responses based on how a person feels.
To achieve this, we propose to use a type of deep learning model called a 
Convolutional Recurrent Neural Network. The idea is that our CRNN combines 
layers that pick up important sound features with layers that track changes 
over time, making it a good fit for picking up patterns in speech.
Moreover, our training data will come from the Toronto Emotional Speech Set. 
This data set has over 2800 audio samples spoken by two voice actresses. 
Before feeding this data, we plan to apply data preprocessing techniques like 
removing background noise and converting the audio to visual spectrograms. Then 
we plan to implement a dynamic learning rate scheduler. This essentially allows 
us to adjust the learning rate of our CRNN during the training process, which 
optimizes our training speed.
Lastly, some extra techniques we plan to use to help generalize speech patterns 
are pitch shifting and time stretching. Pitch shifting changes the tone of the 
sound, making it higher or lower. While time stretching changes how fast or 
slow the sound is but keeps the tone untouched.
\end{abstract}

\end{document}